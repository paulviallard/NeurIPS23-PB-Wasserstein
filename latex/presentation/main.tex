\documentclass{presentation}
\newcommand{\wbf}{\mathbf{w}}
\newcommand{\Wbf}{\mathbf{W}}
\newcommand{\xbf}{\ensuremath{\mathbf{x}}}
\newcommand{\zbf}{\ensuremath{\mathbf{z}}}

\newcommand{\Acal}{\ensuremath{\mathcal{A}}}
\newcommand{\Dcal}{\ensuremath{\mathcal{D}}}
\newcommand{\Fcal}{\ensuremath{\mathcal{F}}}
\newcommand{\Hcal}{\ensuremath{\mathcal{H}}}
\newcommand{\Mcal}{\ensuremath{\mathcal{M}}}
\newcommand{\Ocal}{\ensuremath{\mathcal{O}}}
\newcommand{\Scal}{\ensuremath{\mathcal{S}}}
\newcommand{\Ucal}{\ensuremath{\mathcal{U}}}
\newcommand{\Xcal}{\ensuremath{\mathcal{X}}}
\newcommand{\Ycal}{\ensuremath{\mathcal{Y}}}
\newcommand{\Zcal}{\ensuremath{\mathcal{Z}}}

\newcommand{\Ebb}{\ensuremath{\mathbb{E}}}
\newcommand{\Pbb}{\ensuremath{\mathbb{P}}}
\newcommand{\Rbb}{\ensuremath{\mathbb{R}}}

\newcommand{\Rfrak}{\ensuremath{\mathfrak{R}}}
\newcommand{\Cfrak}{\ensuremath{\mathfrak{C}}}

\newcommand{\Vhat}{\hat{V}}
\newcommand{\Bhat}{\hat{B}}

\newcommand{\LB}{\left[}
\newcommand{\RB}{\right]}
\newcommand{\LC}{\left\{}
\newcommand{\RC}{\right\}}
\defcommand{\RN}{\right\vert}
\newcommand{\LN}{\left\vert}
\newcommand{\LP}{\left(}
\newcommand{\RP}{\right)}

\newcommand{\st}{{\it s.t.}\xspace}
\newcommand{\wrt}{{\it w.r.t.}\xspace}
\newcommand{\aka}{{\it a.k.a.}\xspace}
\newcommand{\eg}{{\it e.g.}\xspace}
\newcommand{\ie}{{\it i.e.}\xspace}
\newcommand{\iid}{{\it i.i.d.}\xspace}

\newcommand{\defeq}{:=}

\DeclareMathOperator*{\EE}{\Ebb}
\DeclareMathOperator*{\PP}{\Pbb}
\DeclareMathOperator*{\argmin}{\mathrm{argmin}}
\DeclareMathOperator*{\vect}{\mathrm{vec}}
\DeclareMathOperator*{\ReLU}{\mathrm{ReLU}}
\newcommand{\KL}{\mathrm{KL}}
\newcommand{\W}{\mathrm{W}}

\newcommand{\AQ}{\alpha}
\newcommand{\BQ}{\beta}
\newcommand{\D}{\mu}
\renewcommand{\H}{\Hcal}
\newcommand{\loss}{\ell}
\renewcommand{\P}{\pi}
\newcommand{\Q}{\rho}
\newcommand{\R}{\Rbb}
\renewcommand{\S}{\Scal}
\newcommand{\Risk}{\text{R}}
\newcommand{\Riskhat}{\hat{\Risk}}
\newcommand{\X}{\Xcal}
\newcommand{\x}{\xbf}
\newcommand{\Y}{\Ycal}
\newcommand{\Z}{\Zcal}
\newcommand{\z}{\zbf}

%%%%%%%%%%%%%%%%%%%%%%%%%%%%%%%%%%%%%%%%%%%%%%%%%%%%%%%%%%%%%%%%%%%%%%%%%%%%%%%

\let\oldQ\Q
\let\oldP\P
\renewcommand{\Q}{\orange{\oldQ}}
\renewcommand{\P}{\green{\oldP}}


\pdfinfo{
  /Title (Learning via Wasserstein-Based High Probability Generalisation Bounds)
  /Author (Paul Viallard)
  /Subject ()

/Keywords ()
}

\title{Learning via Wasserstein-Based High Probability Generalisation Bounds}

\author{\vspace{-1.5cm}

Paul Viallard\textnormal{\textsuperscript{*,1}, Maxime Haddouche\textsuperscript{*,2,3,4}, Umut Şimşekli\textsuperscript{1}, Benjamin Guedj\textsuperscript{2,3}}}

\institute{
\vspace{0.8cm}

\textsuperscript{*} These authors contributed equally to this work\\
\textsuperscript{1} INRIA Paris\\
\textsuperscript{2} INRIA Lille\\
\textsuperscript{3} University College London\\
\textsuperscript{4} Université de Lille\\
}

\date{\vspace{0.5cm}

{\bf NeurIPS 2023}}

\begin{document}

\begin{xplain}
  \titlepage
\end{xplain}

%%%%%%%%%%%%%%%%%%%%%%%%%%%%%%%%%%%%%%%%%%%%%%%%%%%%%%%%%%%%%%%%%%%%%%%%%%%%%%%

\begin{xframe}{(Special case of) PAC-Bayesian learning}

\vspace{-0.5cm}

\begin{block}{}
{\bf PAC-Bayesian learning}\\[0.1cm]
Learning a distribution $\Q$ over models from the data and a prior distribution $\P$
\end{block}

\vspace{-0.2cm}

\begin{figure}
  \includestandalone[width=0.8\linewidth]{figures/intro}
\end{figure}

\vspace{-0.6cm}

\begin{block}{}
{\bf PAC-Bayesian generalisation bounds in a nutshell}\\[0.0cm]
{\footnotesize With probability at least $1-\delta$}\\[-0.7cm]
\begin{align*}
\text{performance gap}(\Q) \le \text{bound}\Big(\text{distance}(\Q, \P), \tfrac{1}{m}, \ln\tfrac{1}{\delta}\Big)
\end{align*}
\vspace{-0.7cm}
\end{block}
\end{xframe}

%%%%%%%%%%%%%%%%%%%%%%%%%%%%%%%%%%%%%%%%%%%%%%%%%%%%%%%%%%%%%%%%%%%%%%%%%%%%%%%


%%%%%%%%%%%%%%%%%%%%%%%%%%%%%%%%%%%%%%%%%%%%%%%%%%%%%%%%%%%%%%%%%%%%%%%%%%%%%%%

\begin{xframe}{About the KL divergence}

  \vspace{0.5cm}

  \begin{figure}
  \includestandalone[width=1.0\linewidth]{figures/KL_inf}
  \end{figure}

  \vspace{-0.5cm}
  \begin{block}{}
  {\bf Drawback:} KL-based PAC-Bayesian bounds (majority of the literature) fail when $\Q$ and $\P$ have disjoint supports \dots
  \end{block}
\end{xframe}

%%%%%%%%%%%%%%%%%%%%%%%%%%%%%%%%%%%%%%%%%%%%%%%%%%%%%%%%%%%%%%%%%%%%%%%%%%%%%%%

\begin{xframe}{About the Wasserstein distance}

  \begin{block}{}
  \centering
  {\bf It is possible to obtain PAC-Bayesian bounds with the Wasserstein distance!}
  \end{block}
  
  \begin{figure}
\includestandalone[width=0.8\linewidth]{figures/ot}
  \end{figure}
\end{xframe}

%%%%%%%%%%%%%%%%%%%%%%%%%%%%%%%%%%%%%%%%%%%%%%%%%%%%%%%%%%%%%%%%%%%%%%%%%%%%%%%

\begin{xframe}{PAC-Bayesian bound for batch learning}

\vfill

\begin{block}{}
{\bf PAC-Bayesian bound for batch learning}~{\scriptsize (special case)}

\vspace{0.2cm}

\begin{itemize}
    \item The loss $\loss$ is $L$-Lipschitz and belongs in $[0,1]$.
    \item The dataset $\S = \{ \z_i \in \Z \}_{i=1}^{m}$ is split into $K$ parts $\S_1,\dots,\S_K$. 
\end{itemize}

\hspace{0.45cm}{\footnotesize With probability at least $1-\delta$}\\[-0.7cm]
\begin{align*}
\underbrace{\EE_{h\sim\Q}\Big[ \EE_{\z\sim\D}\loss(h,\z)- \frac{1}{m}\sum_{i=1}^{m}\loss(h,\z_i) \Big]}_{\text{performance gap}(\Q)} &\le \underbrace{\sum_{i=1}^{K} \frac{2|\S_i|L}{m} \W(\Q, \P_{i,\S})}_{\text{distance}(\Q, \P)} + \underbrace{\sum_{i=1}^{K} \sqrt{\frac{|\S_i|\ln\frac{K}{\delta}}{2m^2}}}_{\text{must be close to $0$}}\\
\text{performance gap}(\Q) &\le \text{bound}\Big(\text{distance}(\Q, \P), \tfrac{1}{m}, \ln\tfrac{1}{\delta}\Big)
\end{align*}

\vspace{-0.2cm}

where $\P_{i,\S}$ {\it does not} depend on $\S_i$.
\end{block}

\vfill

\end{xframe}

%%%%%%%%%%%%%%%%%%%%%%%%%%%%%%%%%%%%%%%%%%%%%%%%%%%%%%%%%%%%%%%%%%%%%%%%%%%%%%%

\begin{xframe}{PAC-Bayesian bound for online learning}

\vfill

\begin{block}{}
{\bf PAC-Bayesian bound for online learning}~{\scriptsize (special case)}


\vspace{0.2cm}

\begin{itemize}
    \item The loss is $\loss$ to be in $[0,1]$ and $L$-Lipschitz
    \item The data arrives sequentially $\S=(\z_i)_{i=1,\dots,m}$
\end{itemize}

\hspace{0.45cm}{\footnotesize With probability at least $1-\delta$}\\[-0.7cm]
\begin{align*}
\underbrace{\frac{1}{m}\sum_{i=1}^m \EE_{h_i\sim \Q_{i}} \Big[\EE[\loss(h_i,\z_i) \mid \Fcal_{i-1}] - \loss(h_i,\z_i) \Big]}_{\text{performance gap}(\Q)}  &\le \underbrace{\frac{2L}{m}\sum_{i=1}^{m}\W(\Q_{i}, \P_{i,\S})}_{\text{distance}(\Q, \P)} + \underbrace{\sqrt{\frac{2\ln\LP\frac{1}{\delta}\RP}{m}}}_{\text{close to $0$}}\\
\text{performance gap}(\Q) &\le \text{bound}\Big(\text{distance}(\Q, \P), \tfrac{1}{m}, \ln\tfrac{1}{\delta}\Big)
\end{align*}

where $\P_{i,\S}$ {\it does not} depend on $(\z_j)_{j=i,\dots,m}$.
\end{block}

\vfill

\end{xframe}

%%%%%%%%%%%%%%%%%%%%%%%%%%%%%%%%%%%%%%%%%%%%%%%%%%%%%%%%%%%%%%%%%%%%%%%%%%%%%%%

\begin{xframe}{Other contributions in the paper}

\vfill

\begin{block}{}
{\bf Other contributions in the paper:}\\[0.5cm]
\begin{itemize}
    \item PAC-Bayesian bounds for heavy-tailed losses (possibly unbounded)\\[0.5cm]
    \item Derive new learning algorithms (batch and online learning)\\[0.2cm]
    \item[] $\hookrightarrow$ Inspired from our theoretical bounds\\[0.2cm]
    \item[] $\hookrightarrow$ Better generalisation than ERM and classical regularisations 
\end{itemize}

\end{block}

\vfill

\end{xframe}

%%%%%%%%%%%%%%%%%%%%%%%%%%%%%%%%%%%%%%%%%%%%%%%%%%%%%%%%%%%%%%%%%%%%%%%%%%%%%%%

\appendix

\begin{xtitle}

\vspace{2.0cm}
{\bf Thank you for your attention!}\\

\vspace{1.0cm}
{\bf Link to the paper:}\\
\href{https://arxiv.org/abs/2306.04375}{{\tt https://arxiv.org/abs/2306.04375}}\\

\vfill

\end{xtitle}

%%%%%%%%%%%%%%%%%%%%%%%%%%%%%%%%%%%%%%%%%%%%%%%%%%%%%%%%%%%%%%%%%%%%%%%%%%%%%%%
 
\end{document}
